Künstliche Intelligenz kurz KI genannt (eng. artificial intelligence kurz AI), it eine Teilgebiet der Informatik. KI beschäftigt sich mit der Automatisierung intelligentes Verhalten und maschinelles Lernen. Die Definition des Begriffs \textit{künstliche Intelligenz} ist nicht ganz einfach, da die Definition von \textit{Intelligenz} mangelhaft ist.\vspace{0.2cm}

Im folgenden zwei Versuche KI zu definieren. In \cite{def_ai_weber}, Seite 14 wird KI definiert als ,,Künstliche Intelligenz ist die Eigenschaft eines IT-Systems, \textit{menschenähnliche}, intelligente Verhaltensweisen zu zeigen.'' und in einem Betrag von \cite{def_ai_ronsdorf} als ,,Unter künstlicher Intelligenz (KI) verstehen wir Technologien, die menschliche Fähigkeiten im Sehen, Hören, Analysieren, Entscheiden und Handeln ergänzen und stärken.''\vspace{0.2cm}

Bei künstlicher Intelligenz werden zwei Arten unterschieden. Die \textit{Starke KI}, bei der die System wie Menschen interagieren und dabei lernen Zusammenhänge herzustellen und eigene Entscheidungen zu treffen. Diese Art der künstlichen Intelligenz existiert zurzeit nicht.\vspace{0.2cm}

Eine weitere Art ist die \textit{Schwache KI}. Hierbei geht es nicht darum, dass Maschinen ein eigenes Bewusstsein entwickeln, sondern, das Entscheidungen auf der Grundlage von Mathematischen Methoden getroffen werden. Diese Art der künstlichen Intelligenz hat in den letzten Jahren bedeutende Fortschritte erzielt. So auch im Bereich des Online Handels.