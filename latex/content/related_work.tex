Xia Song at all beschäftigten sich bereit 2019 im Journal \cite{Song_2019} mit dem Thema künstliche Intelligenz in Onlinehandel. Hier wird der aktuelle Stand der künstlichen Intelligenz im Onlinehandel untersucht. Insbesondere die Aspekte der intelligenten Logistik und Empfehlungssystemen. Ebenso wie Xia Song at all setzt sich auch \cite{Soni_2020} mit künstlicher Intelligenz im Onlinehandel auseinander. Der Fokus liegt hier auf verwendete Systeme zur Unterstützung des Onlinehandels.\vspace{0.2cm}

Ebenfalls mit Empfehlungssystemen beschäftigt sich \cite{Abdul_Hussien_2021} in seinem Paper aus dem Jahr 2021. Er untersucht insbesondere den \textit{Collaborative Filtering Algorithm} (CF).\vspace{0.2cm}

Mit dem Sammeln, Auswerten und Anzeigen von Daten beschäftigt sich \cite{Keerthana_2021}. In dem Paper wird gezeigt, wie Daten aus Kundenbewertungen gesammelt und aufbereitet werden, um daraus die Meinungen der Nutzer grafisch darzustellen.\vspace{0.2cm}

Mit der Steigerung der Kundenzufriedenheit beschäftigte sich \cite{Wong_2020}. Es werden die zwei wichtigsten Faktoren aufgezeigt, welche die Kundenzufriedenheit beeinflussen. Dazu wurden verschiedene Algorithmen evaluiert, um die bestmögliche Genauigkeit zu erhalten, nach welchen Kriterien die Kundenzufriedenheit gesteigert werden konnte. Ebenso wie in \cite{Keerthana_2021} gab es auch hier Datensätze, die unausgeglichen, verzerrt oder fehlten. Diese wurde auch hier bereinigt.\vspace{0.2cm}

Ein weiterer wichtiger Punkt im Onlinehandel ist die Logistik. In \cite{Lv_2021} untersucht Jian Lv Algorithmen der künstlichen Intelligenz zur Verbesserung der Bedarfsermittlung und Optimierung des Lagerbestandes. Die Untersuchung basiert auf Daten der chinesischen Logistik-Branche.\vspace{0.2cm}

Auf die Analyse von FAQ-System zielt das Paper \cite{Pan_2018} ab. Es werden Modelle zur Erkennung der Benutzerabsicht, ein Muster des \textit{Association Rule Mining} Modell und ein automatisches E-Commerce-Antwortsystem analysiert.
