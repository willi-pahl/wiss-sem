Optimierung der Ressourcen ist ein wichtiger Teil, um im hart umkämpften Markt der Onlinehändler konkurrenzfähig zu bleiben. Um eine Reduzierung des Ressourcenverbrauchs zu bewirken, müssen die Prozesse im Onlinehandel analysiert und automatisiert werden. Künstliche Intelligenz kann nicht nur einfache Aufgaben übernehmen und die Ressourcen einsparen, sondern senkt auch die Fehlerquote bei der Abarbeitung dieser Aufgaben.\vspace{0.2cm}

Sollen die Prozesse optimal erfasst werden, ist es essenziell, dass die Mitarbeiter des Onlinehandels frühzeitig in die Prozessoptimierung eingebunden und ihnen klar wird, dass sie nicht überflüssig werden, sondern an anderen Stellen im Handel eingesetzt werden. Wir betrachten ein Beispiel einer E-Commerce-Plattform, die neben dem eigentlichen Shop über einen eigenen Lieferservice und über eigene Abteilungen zur Vermarktung und Kundenbetreuung verfügt. Diese Plattform hat künstliche Intelligenz für die Prozessoptimierung eingeführt.\vspace{0.2cm}

Ich werde die Erhebung und Verwendung der Daten bewerten, die für den Einsatz von künstlicher Intelligenz erforderlich sind, um die Prozesse zu optimieren. Ebenfalls wird eine mögliche Systemlandschaft vorgestellt, um die erforderliche Infrastruktur bereit zustellen. In dieser Systemlandschaft wird die künstliche Intelligenz optimal eingesetzt und eine Optimierung einzelner Komponenten kann nach Bedarf erfolgen.\vspace{0.2cm}

Ich betrachte in dieser Arbeit drei Bereiche \textit{Planung und Prozess}, \textit{Produktangebot und -darstellung} sowie \textit{Beratung und Service}. In alles drei Bereichen werden Lösungen aufgezeigt, die die Prozesse teilweise oder voll automatisieren und somit zur Ressourcenoptimierung beitragen. Der Bereich \textit{Planung und Prozess} zeigt Lösungen, die Prozesse des E-Commerce optimiert. Hier werden Technologien eingesetzt, die Prozesse optimieren, die neben dem eigentlichen Onlinegeschäft liegen. Dies sind Prozesse der Bedarfsermittlung, Lagerverwaltung und Liefermanagement. Prozesse, die für die Kundenneugewinnung optimiert werden, finden sich im Bereich der \textit{Produktangebote und -darstellung} wieder. Die hier eingesetzten Technologien, die mithilfe der künstlichen Intelligenz gesteuert und überwacht werden, übernehmen Aufgaben der Webseiten- und Produktgestaltung. Ebenso werden Methoden zur Vorhersage und optimierten Anzeigen möglicher benötigter Produkte verwendet. Bleibt noch der Bereich \textit{Beratung und Service}. Die hier eingesetzten Methodiken und Technologien unterstützen Prozesse zur Kundengewinnung und -betreuung. Auch Prozesse zur Kundenrückgewinnung werden betrachtet. Alle Optimierungen für das Kundenmanagement unterstützen dabei, die Arbeitsaufwände der Mitarbeiter zu reduzieren. Somit werden Mitarbeiterressourcen freigesetzt, die beispielsweise im Kundendienst anspruchsvollere Aufgaben übernehmen können.\vspace{0.2cm}

Um die Optimierung zu demonstrieren, werden folgende Lösungen gezeigt,

\begin{enumerate}[label=(\arabic*)]
	\item Es wird gezeigt wie künstliche Intelligenz dazu beträgt die Ressourcen im Onlinehandel einzusparen.
	\item Ich schlage Technologien vor, welche die Kundenzufriedenheit steigern und somit zur Umsatzsteigerung beitragen.
	\item Es wird gezeigt, dass durch den Einsatz von künstlicher Intelligenz die Mitarbeiterzufriedenheit steigt, weil diese jetzt anspruchsvollere Aufgaben übernehmen und durch erhöhte Motivation zur Umsatzsteigerung beitragen.
\end{enumerate}