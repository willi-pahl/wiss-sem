Eine künstliche Intelligenz kann anhand von aktuellen Nutzerdaten und Statistiken von gesammelten Daten viele Aufgaben aus dem Bereich des Online-Handels übernehmen. Hinzu kommen äußere Faktoren wie Tageszeit, Tage vor Feiertage, Geographische Lage, aktuelle Trends oder Jahreszeit\vspace{0.2cm}

Bereits bei der Generierung der Produktpräsentation kann die künstliche Intelligenz personalisierte Produkte  bereitstellen und den Aufbau des Onlineshop an das Verhalten des Nutzers anpassen. Somit wird das gesamte ,,Customer Journey'' an die Nutzerbedürfnisse angepasst und nicht relevante Elemente können entfallen. Hier gilt das Sprichwort ,,Weniger ist oft mehr''.\vspace{0.2cm}

Durch künstliche Intelligenz wird auch der Kundendienst entlastet. Hier helfen dem Nutzer Chatbots und visualisierte Suche bei der Kaufentscheidung. Hierbei lässt sich eine Reduzierung im Kundendienst beobachten.\vspace{0.2cm}

Weiterhin kann künstliche Intelligenz Vorhersagen zu Verkäufen tätigen und somit den optimalen Lagerbestand ermittelt. So können nicht nur Lagerhaltungs-, Marketing-, Personal- und Retouren Kosten eingespart, sondern auch die Umsätze werden nachhaltig gesteigert.\vspace{0.2cm}

In \cite{glaess2018kuenstliche} auf Seite 3 werden die Einsatzgebiete von künstlicher Intelligenz unterteilt in Planung \& Prozess, Produktangebot \& -darstellung und Beratung \& Service.
