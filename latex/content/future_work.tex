Diese Arbeit soll einen Einblick in die Verwendung von künstlicher Intelligenz im Onlinehandel geben. Hier werden nicht alle Möglichkeiten, die künstliche Intelligenz bieten, diskutiert und analysiert. Somit gibt es sehr viel Spielraum für die Betrachtung weitere Komponenten, die im Onlinehandel eingesetzt werden können. Gerade mittelgroße Onlinehändler könnten schön, mit einem Teil der hier beschriebenen Komponenten ihre Ressourcen optimieren.\vspace{0.2cm}

Da die Umsätze im Onlinehandel in den nächsten Jahren anwachsen werden, ist auch eine wirtschaftliche Kraft hinter der Entwicklung der künstlichen Intelligenz im Onlinehandel vorhanden. So dass in deren Entwicklung viel Know-how und Kapitel fließen wird. Immer mehr Online Plattformen werden die künstliche Intelligenz einsetzen, so dass hier ein wachsender Markt entsteht.\vspace{0.2cm}

So werden in naher Zukunft auch fertige Lösungen für eine Integration der künstlichen Intelligenz in Onlinehandel Plattformen entstehen, die dessen Einsatz einfach und effizient gestalten werden.\vspace{0.2cm}

Eine weitere gesellschaftliche Herausforderung ist die Frage nach den gesetzlichen und ethischen Grenzen der künstliche Intelligenz. Politik und Wissenschaft sollten frühzeitig die Rahmenbedingungen für künstliche Intelligenz im Onlinehandel schaffen. Es sollte nicht der Wirtschaft überlassen werden, wie weit der Eingriff in die menschliche Privatsphäre durch künstliche Intelligenz erfolgen darf.
